\section{Espacios métricos}

\begin{definition}
    Un \emph{espacio métrico} es un par $(X, d)$, donde $X$ es un conjunto y $d: X \times X \to \mathbb{R}_{\geq 0}$ una función llamada \emph{distancia} (o \emph{métrica}), que satisface las siguientes propiedades para todo $x, y, z \in X$:
    \begin{enumerate}
        \item $d(x, y) = d(y, x)$.
        \item $d(x, z) \le d(x, y) + d(y, z)$.
        \item $d(x, y) = 0$ si y sólo si $x = y$.
    \end{enumerate}
\end{definition}

Una función $d: X \times X \to \mathbb{R}_{\geq 0}$ es una pseudo-métrica si cumple la simetría, la desigualdad triangular y si además cumple que
\begin{center}
    si $x = y$, entonces $d(x, y) = 0$.
\end{center}

\begin{remark}
    Como se cumple la desigualdad triangular, también se cumple
    \begin{itemize}
        \item $d(x_{1}, x_{n}) \leq d(x_{1}, x_{2}) + d(x_{2}, x_{3}) + \dots + d(x_{n-1}, x_{n})$.
        \item $\lvert d(x, z) - d(y, z) \rvert \leq d(x, y)$.
    \end{itemize}
\end{remark}

Veamos algunos ejemplos de espacios métricos.

\begin{example}
\begin{itemize}
    \item Para cualquier conjunto $X$, la \textbf{\textit{métrica discreta}} está definida por $d(x, y) = \delta_{xy} = \begin{cases} 0 & \text{si } x = y \\ 1 & \text{si } x \neq y \end{cases}$.

    \item Si $(X, d)$ es un espacio métrico, entonces $d'(x, y) = \min(d(x, y), 1)$ también es una métrica en $X$, llamada métrica acotada equivalente.

    \item Si $(V, \langle \cdot, \cdot \rangle)$ es un espacio vectorial con producto interno (sobre $\mathbb{R}$ o $\mathbb{C}$), entonces $\|x\| = \sqrt{\langle x, x \rangle}$ es una norma, y $d(x, y) = \|x - y\|$ es una métrica inducida por la norma.

    \item El espacio $C([a, b], \mathbb{R})$ de funciones continuas $f: [a, b] \to \mathbb{R}$. Junto con la norma
    \begin{itemize}
        \item $L^2$: $\|f\|_2 = \left( \int_a^b |f(x)|^2 dx \right)^{1/2}$.
        \item $\|f\|_{\infty} = \sup_{x \in [a,b]} |f(x)|$.
    \end{itemize}
\end{itemize}
\end{example}

\begin{definition}
    Definimos
    $$
        B(x, r) = \set{ y \in X \mid d(x, y) < r}
    $$ como la \emph{bola abierta} centrada en $x$ con radio $r$. Análogamente, la \emph{bola cerrada} es 
    $$
        \overline{B}(x, r) = \set{ y \in X \mid d(x, y) \leq  r}.
    $$
\end{definition}

Esto nos lleva a la definición de entorno.

\begin{definition}
    Un \emph{entorno} de $x \in X$ es un subconjunto $V \subseteq X$ tal que $x \in V$ y existe una bola $B(x, r) \subseteq V$.
\end{definition}

\begin{remark}
    Notemos que $B(x, r)$ siempre es un entorno de $x$.
\end{remark}

Damos la definición de alguna terminología que utilizaremos más adelante.

\begin{definition}
    Sea $A \subseteq X$. Definimos
    \begin{itemize}
        \item El \emph{interior} de $A$ como 
        $$
            A^{\degree} = \set{ x \in X \mid \exists r > 0 \text{ tal que } B(x, r) \subseteq A}.
        $$
        \item La \emph{clausura} de $A$ como
        $$
            \overline{A} = \set{ x \in X \mid \forall r > 0, B(x, r) \cap A \neq \emptyset}.
        $$
        \item La \emph{frontera} de $A$ como
        $$
            \partial A = \overline{A} - A^{\degree}.
        $$
        \item El \emph{exterior} de $A$ como
        $$
            \operatorname{ext} A = (X - A)^{\degree}.
        $$
    \end{itemize}
    Además, 
    \begin{itemize}
        \item Si $A = A^{\degree}$, entonces decimos que $A$ es \emph{abierto}.
        \item Si $A = \overline{A}$, entonces decimos que $A$ es \emph{cerrado}.
    \end{itemize}
\end{definition}

Definimos dos términos relacionados a la distancia.

\begin{definition}
    Sea $A \subseteq X$. Definimos el \emph{diámetro} de $A$ como
    $$
        \operatorname{diam}(A) = \sup_{a, b \in A} d(a, b).
    $$
\end{definition}

Y la distancia entre un punto y un conjunto.

\begin{definition}
    Sea $x \in X$ y $A \subseteq X$. Definimos la \emph{distancia} entre $x$ y $A$ como
    $$
        d(x, A) = \inf_{a \in A} d(x, a).
    $$
\end{definition}

\subsection{Sucesiones convergentes}

\begin{definition}
    Sea $(x_n)_{n \in \mathbb{N}}$ una sucesión en $X$. Decimos que $\lim_{n \to \infty} x_{n} = x$ (o $x_{n} \xrightarrow[n \to \infty]{} x$) si para todo $\varepsilon > 0$, existe $N \in \mathbb{N}$ tal que $n \geq N$ implica $d(x_n, x) < \varepsilon$.
\end{definition}

\begin{remark}
    Es equivalente tomar $d(x_n, x) \leq \varepsilon$.
\end{remark}

\begin{proposition}
    Sea $(x_n)_{n \in \mathbb{N}}$ una sucesión en $X$. Si 
    $$
        \lim_{n \to \infty} x_n = x \quad\text{y}\quad \lim_{n \to \infty} x_n = y,
    $$
    entonces $x = y$.
\end{proposition}

\begin{proof}
    Sea $\varepsilon > 0$. Entonces, existe $N \in \mathbb{N}$ tal que 
    $$
        d(x_n, x) \leq \frac{\varepsilon}{2} \quad \text{y} \quad d(x_n, y) \leq \frac{\varepsilon}{2},
    $$
    para todo $n \geq N$. Por lo tanto,
    \begin{align*}
        0 \leq  d(x, y) \leq d(x, x_n) + d(x_n, y) < \varepsilon.
    \end{align*}
    Entonces, $d(x, y) = 0$ lo que implica que $x = y$.
\end{proof}

\begin{definition}
    Una sucesión $(x_n)_{n \in \mathbb{N}}$ en $X$ es una \emph{sucesión de Cauchy} si para todo $\varepsilon > 0$, existe $N \in \mathbb{N}$ tal que $n, m \geq N$ implica $d(x_n, x_m) < \varepsilon$.
\end{definition}

\begin{proposition}
    Sea $(x_n)_{n \in \mathbb{N}}$ una sucesión en $X$. Si $(x_n)$ converge, entonces es de Cauchy.
\end{proposition}

\begin{proof}
    Sea $\lim_{n \to \infty} x_n = x$ y sea $\varepsilon > 0$. Por definición de límite, existe $N \in \mathbb{N}$ tal que $d(x_n, x) < \frac{\varepsilon}{2}$, para todo $n \geq N$. Entonces,
    $$
        d(x_n, x_m) \leq d(x_n, x) + d(x, x_m) < \frac{\varepsilon}{2} + \frac{\varepsilon}{2} = \varepsilon.
    $$
    Por lo tanto, $(x_n)$ es una sucesión de Cauchy.
\end{proof}

\begin{definition}
    Un espacio métrico $(X, d)$ se dice \emph{completo} si toda sucesión de Cauchy tiene límite en $X$.   
\end{definition}

\begin{example}
    Sea $(X, \delta)$ un espacio métrico con la métrica discreta. Entonces, $(X, \delta)$ es completo.
\end{example}

\begin{proof}[Solución]
    Toda sucesión de Cauchy en $(X, \delta)$ es eventualmente constante. Por lo tanto, converge a un elemento de $X$.
\end{proof}